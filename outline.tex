\documentclass[12pt]{article}

\addtolength{\textwidth}{1.4in}
\addtolength{\oddsidemargin}{-.7in} %left margin
\addtolength{\evensidemargin}{-.7in}
\setlength{\textheight}{8.5in}
\setlength{\topmargin}{0.0in}
\setlength{\headsep}{0.0in}
\setlength{\headheight}{0.0in}
\setlength{\footskip}{.5in}
\renewcommand{\baselinestretch}{1.0}
\setlength{\parindent}{0pt}
\linespread{1.1}

\usepackage[pdftex,
bookmarks=true,
bookmarksnumbered=false,
pdfview=fitH,
bookmarksopen=true]{hyperref}

\usepackage{amssymb, amsmath, amsthm, bm}
\usepackage{graphicx,csquotes,verbatim}
\usepackage[backend=biber,block=space,style=authoryear]{biblatex}
\setlength{\bibitemsep}{\baselineskip}
\usepackage[american]{babel}
%dell laptop
\addbibresource{C:/Users/Kristy/Dropbox/Research/xBibs/tradeagreements.bib}
%\addbibresource{C:/Users/Kristy/Documents/Dropbox/Research/xBibs/tradeagreements.bib}
\renewcommand{\newunitpunct}{,}
\renewbibmacro{in:}{}


\DeclareMathOperator*{\argmax}{arg\,max}
\usepackage{xcolor}
\hbadness=10000

\newtheorem{proposition}{Proposition}
\newcommand{\ve}{\varepsilon}
\newcommand{\ov}{\overline}
\newcommand{\un}{\underline}
\newcommand{\ta}{\theta}
\newcommand{\al}{\alpha}
\newcommand{\Ta}{\Theta}
\newcommand{\expect}{\mathbb{E}}
\newcommand{\Bt}{B(\bm{\tau^a})}
\newcommand{\bta}{\bm{\tau^a}}
\newcommand{\btn}{\bm{\tau^n}}
\newcommand{\btw}{\bm{\tau^{tw}}}
\newcommand{\ga}{\gamma}
\newcommand{\Ga}{\Gamma}
\newcommand{\de}{\delta}

\begin{document}
\begin{center}
  Preliminary Outline for Objective Function Project
\end{center}

\vskip.3in
\section{Motivation}
Lots of governments don't seem to be maximizing either social welfare, a weighted sum of welfare and contributions, or Baldwin-style government objective function. Those all imply that governments value consumer surplus to some degree. 
\begin{itemize}
	\item It may be that Chinese government's objective is to maximize domestic employment (Caroline Freund)
	\item Not clear exactly what Trump is maximizing. He talks about employment, trade balance. More likely profits? Tariff revenue? Could be median voter sentiment?
		\begin{itemize}
			\item What does a simple mercantilist objective function look like?
			\item How does median voter's sentiment actually get translated into trade policy? Read John McLaren's new paper
		\end{itemize}
	\item Emanuel: ``winning.'' Is this share of surplus? Or just share.
		\begin{itemize}
			\item Trump wants to 'get a better deal.' What does this mean? More market access?
		\end{itemize}
\end{itemize}


\vskip1in
\section{Main Idea}
What a trade agreement is \textit{for} depends on what governments care about.
\begin{itemize}
	\item 
\end{itemize}


\vskip1in
\section{Model}
Need trade agreement model, two countries
\begin{itemize}
	\item Probably shouldn't be a balanced-trade model
	\item Probably should be asymmetric
		\begin{itemize}
			\item Certainly bargaining power should be asymmetric. But where does it come from?
			\item Market size? Technology? Industrial structure?
			\item Hold-up style outside option in terms of supply/demand?
		\end{itemize}
\end{itemize}
 
\vskip.5in
\begin{itemize}
	\item I.O. bilateral bargaining models have voluntary exchange. How does this vary under Trump-style threats?
		\begin{itemize}
			\item No rules?
			\item Are participation constraints asymmetric? Is it just the outside option that varies? The ability to punish varies for sure. Of course how open the economies are plays into both, along with institutional constraints (both domestic and international).
		\end{itemize}
\end{itemize}

		
\end{document}